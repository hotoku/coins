\documentclass{jsarticle}

\newcommand{\numput}{T}
\newcommand{\maisu}{M}
\newcommand{\poisson}{\mathrm{Poisson}}
\newcommand{\categorical}{\mathrm{Categorical}}
\newcommand{\meannumput}{\mu}
\newcommand{\ncoins}[1]{N^{#1}}
\newcommand{\weight}[1]{W^{#1}}
\newcommand{\weightall}{W}
\newcommand{\observation}{X}
\newcommand{\ampeq}{&=&}
\newcommand{\weightobs}{W_{\mathrm{obs}}}
\newcommand{\expectation}{\mathrm{E}}
\newcommand{\weightvar}{\beta}

\begin{document}
\section{生成モデル}
コインの重量の生成モデルを考える。貯金箱へのコインの投入が$\numput$回あったとする。
第$n$回目の投入で、$\maisu_{n}$枚のコインを投入したとする。
$\numput, \maisu_{n}$の事前分布を
\begin{eqnarray*}
 \numput | \meannumput &\sim& \poisson(\meannumput) \\
 \maisu_{n} | \numput &\sim& \categorical\left(\frac{1}{999}\right)
\end{eqnarray*}
とする。$\maisu = (\maisu_{1}, \maisu_{2}, \ldots, \maisu_{\numput})$とする。
$n$回目の投入での$p$円硬貨の枚数を$\ncoins{p}_{n}$とする($p\in\{1,5,10,50,100,500\}$)。
$\ncoins{p}_{n}$は、合計金額が$\maisu_{n}$となるような最小枚数の組み合わせとする。
$p$円硬貨の重さを$\weight{p}$とする。
貯金箱の中身全体の重さ$\weightall$は
\begin{eqnarray*}
 \weightall \ampeq \sum_{n=1}^{\numput} \sum_{p} \ncoins{p}_{n} \weight{p}
\end{eqnarray*}
と計算できる。重さの観測値$\weightobs$は、$\weightall$を期待値とするガンマ分布から発生するとする
\begin{eqnarray*}
 \weightobs | \maisu, \weightvar \sim \Gamma(\weightall \weightvar, \weightvar)
\end{eqnarray*}
ただし、$\Gamma(k, \lambda)$の密度関数は、
\begin{eqnarray*}
 f(x; k, \lambda) \ampeq \frac{\lambda^{k}}{\Gamma(k)}x^{k-1}\exp(-\lambda x)
\end{eqnarray*}
で与えられる。この分布の期待値は
\begin{eqnarray*}
 \expectation [X] = \frac{k}{\lambda}
\end{eqnarray*}
である。$k=\weightall \weightvar, \lambda=\weightvar$と置くと、$\expectation[\weightobs] = \weightall$となっていることが確認できる。

正のパラメータ$\meannumput, \weightvar$の事前分布は、十分に大きい分散を持つガンマ分布に従うとする。

\section{同時分布}
全確率変数の同時分布は
\begin{eqnarray*}
 f(\meannumput, \numput, \maisu, \weightobs) \ampeq f(\weightobs | \maisu, \weightvar)f(\maisu | \numput)f(\numput | \meannumput)f(\meannumput)f(\weightvar) \\
  \ampeq \Gamma(\weightobs | \weightall \weightvar, \weightvar) \left(\frac{1}{999}\right)^{\numput} \poisson(T | \meannumput) \Gamma(\weightvar | \alpha_{1}, \alpha_{2}) \Gamma(\meannumput | \alpha_{3}, \alpha_{4})
\end{eqnarray*}
で計算できる。$\alpha_{i}, i=1,2,3,4$は、適当なハイパーパラメータである。


\section{サンプリング法}
サンプリングはメトロポリス・ヘイスティング法で行う。アルゴリズムの導出はWikipediaを参考にした。
サンプリングが必要なパラメータは、$\meannumput, \numput, \maisu, \weightvar$の3+$\numput$個である。
$\theta = (\meannumput, \numput, \maisu, \weightvar)$として、現在のパラメータを$\theta$、提案されたパラメータを$\theta'$と書く。

提案分布$g(\theta\rightarrow\theta')$を、次のように与える。
\begin{enumerate}
 \item $\meannumput, \weightvar$は、現在の値を期待値とし、適当な分散を持つガンマ分布
 \item $\numput$は、現在の値を期待値とするポアソン分布
 \item $\maisu$は、$1\sim 999$の一様分布
\end{enumerate}

提案分布の密度関数は、次で与えられる。
\begin{eqnarray*}
 g(\theta\rightarrow\theta') \ampeq \left(\frac{1}{999}\right)^{\numput'} \poisson(\numput' | \numput) \Gamma(\meannumput' | \meannumput \alpha_{5}, \alpha_{5}) \Gamma(\weightvar' | \weightvar \alpha_{6}, \alpha_{6})
\end{eqnarray*}

採択率$A(\theta\rightarrow\theta')$は、
\begin{eqnarray*}
 
\end{eqnarray*}

\end{document}
