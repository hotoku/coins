\documentclass{jsarticle}

\newcommand{\numput}{T}
\newcommand{\maisu}{M}
\newcommand{\poisson}{\mathrm{Poisson}}
\newcommand{\categorical}{\mathrm{Categorical}}
\newcommand{\meannumput}{\mu}
\newcommand{\ncoins}[1]{N^{#1}}
\newcommand{\weight}[1]{W^{#1}}
\newcommand{\weightall}{W}
\newcommand{\observation}{X}
\newcommand{\ampeq}{&=&}
\newcommand{\weightobs}{W_{\mathrm{obs}}}
\newcommand{\expectation}{\mathrm{E}}
\newcommand{\variance}{V}
\newcommand{\weightsd}{\beta}
\newcommand{\gammak}{k}
\newcommand{\gammal}{\lambda}
\newcommand{\parmean}{m}
\newcommand{\parsd}{s}

\begin{document}
\section{生成モデル}
コインの重量の生成モデルを考える。貯金箱へのコインの投入が$\numput$回あったとする。
第$n$回目の投入で、$\maisu_{n}$枚のコインを投入したとする。
$\numput, \maisu_{n}$の事前分布を
\begin{eqnarray*}
 \numput | \meannumput &\sim& \poisson(\meannumput) \\
 \maisu_{n} | \numput &\sim& \categorical\left(\frac{1}{999}\right)
\end{eqnarray*}
とする。$\maisu = (\maisu_{1}, \maisu_{2}, \ldots, \maisu_{\numput})$とする。
$n$回目の投入での$p$円硬貨の枚数を$\ncoins{p}_{n}$とする($p\in\{1,5,10,50,100,500\}$)。
$\ncoins{p}_{n}$は、合計金額が$\maisu_{n}$となるような最小枚数の組み合わせとする。
$p$円硬貨の重さを$\weight{p}$とする。
貯金箱の中身全体の重さ$\weightall$は
\begin{eqnarray*}
 \weightall \ampeq \sum_{n=1}^{\numput} \sum_{p} \ncoins{p}_{n} \weight{p}
\end{eqnarray*}
と計算できる。重さの観測値$\weightobs$は、$\weightall, \weightsd$を期待値、標準偏差とするガンマ分布から発生するとする。
\begin{eqnarray*}
 \weightobs | \maisu, \weightsd &\sim& \Gamma \\
 \expectation[\weightobs] \ampeq \weightall
\end{eqnarray*}

正のパラメータ$\meannumput, \weightsd$の事前分布は、十分に大きい分散を持つガンマ分布に従うとする。

\subsection{ガンマ分布について}
$\Gamma(\gammak, \gammal)$の密度関数は、
\begin{eqnarray*}
 f(x; \gammak, \gammal) \ampeq \frac{\gammal^{\gammak}}{\Gamma(\gammak)}x^{\gammak-1}\exp(-\gammal x)
\end{eqnarray*}
で与えられる。この分布の期待値と分散は
\begin{eqnarray*}
 \expectation [X] \ampeq \frac{\gammak}{\gammal} \\
 \variance [X] \ampeq \frac{\gammak}{\gammal^{2}} \\
\end{eqnarray*}
逆に、平均と分散が与えられる場合には、
\begin{eqnarray*}
 \gammal \ampeq \frac{\expectation [X]}{\variance [X]} \\
 \gammak \ampeq \frac{\expectation [X]^{2}}{\variance [X]}
\end{eqnarray*}
と計算できる。

である。 $\weightobs$の観測分布のパラメータ$\gammak_{\weightobs}, \gammal_{\weightobs}$は、
\begin{eqnarray*}
 \gammal_{\weightobs} \ampeq \frac{W}{\weightsd^{2}} \\
 \gammak_{\weightobs} \ampeq \frac{W^{2}}{\weightsd^{2}} \\
\end{eqnarray*}



\section{同時分布}
全確率変数の同時分布は
\begin{eqnarray*}
 f(\meannumput, \numput, \maisu, \weightobs) \ampeq f(\weightobs | \maisu, \weightsd)f(\maisu | \numput)f(\numput | \meannumput)f(\meannumput)f(\weightsd) \\
 \ampeq \Gamma(\weightobs | \weightall \weightsd, \weightsd) \left(\frac{1}{999}\right)^{\numput} \poisson(T | \meannumput) 
  \Gamma(\weightsd | \gammak_{\weightsd}, \gammal_{\weightsd}) \Gamma(\meannumput | \gammak_{\meannumput}, \gammal_{\meannumput})
\end{eqnarray*}
と書ける。$\gammak_{\weightsd}, \gammal_{\weightsd}, \gammak_{\meannumput}, \gammal_{\meannumput}$はハイパーパラメータであるが、この形だと大きさが分かりづらいので、
$\parmean_{\weightsd}, \parsd_{\weightsd}, \parmean_{\meannumput}, \parsd_{\meannumput}$をハイパーパラメータとする(各パラメータの平均と標準偏差)。
すると、
\begin{eqnarray*}
 \parmean_{\weightsd} \ampeq \frac{\gammak_{\weightsd}}{\gammal_{\weightsd}} \\
 \parsd_{\weightsd}^{2} \ampeq \frac{\gammak_{\weightsd}}{\gammal_{\weightsd}^{2}}
\end{eqnarray*}
より、
\begin{eqnarray*}
 \gammal_{\weightsd} \ampeq \frac{\parmean_{\weightsd}}{\parsd_{\weightsd}^{2}} \\
 \gammak_{\weightsd} \ampeq \parmean_{\weightsd} \gammal_{\weightsd} = \frac{\parmean_{\weightsd}^{2}}{\parsd_{\weightsd}^{2}}
\end{eqnarray*}
となる。

\section{サンプリング法}
サンプリングはメトロポリス・ヘイスティング法で行う。アルゴリズムの導出はWikipediaを参考にした。
サンプリングが必要なパラメータは、$\meannumput, \numput, \maisu, \weightsd$の3+$\numput$個である。
$\theta = (\meannumput, \numput, \maisu, \weightsd)$として、現在のパラメータを$\theta$、提案されたパラメータを$\theta'$と書く。

提案分布$g(\theta\rightarrow\theta')$を、次のように与える。
\begin{enumerate}
 \item $\meannumput, \weightsd$は、現在の値を期待値とし、適当な分散を持つガンマ分布
 \item $\numput$は、現在の値を期待値とするポアソン分布
 \item $\maisu$は、$1\sim 999$の一様分布
\end{enumerate}

提案分布の密度関数は、次で与えられる。
\begin{eqnarray*}
 g(\theta\rightarrow\theta') \ampeq \left(\frac{1}{999}\right)^{\numput'} \poisson(\numput' | \numput) 
  \Gamma(\meannumput' | \gammak'_{\meannumput}, \gammal'_{\meannumput}) \Gamma(\weightsd' | \gammak'_{\weightsd}, \gammal'_{\weightsd})
\end{eqnarray*}
ここで、$\meannumput', \weightsd'$の提案分布の標準偏差をハイパーパラメータ$\parsd'_{\meannumput}, \parsd'_{\weightsd}$として固定しておく。
これらの提案分布のパラメータは、
\begin{eqnarray*}\\
 \gammal'_{\meannumput} \ampeq \frac{\meannumput}{(\parsd'_{\meannumput})^{2}} \\
 \gammak'_{\meannumput} \ampeq \frac{\meannumput^{2}}{(\parsd'_{\meannumput})^{2}}
\end{eqnarray*}
などと計算できる。

\section{logsumexp}
\[
 \exp(\sum_{i} a_{i}) = \exp(\sum_{i}^{N} (a_{i} - \min(a_{j})) + N\min (a_{j}))
\]

\end{document}
